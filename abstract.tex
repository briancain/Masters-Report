% +--------------------------------------------------------------------+
% | Abstract Page
% +--------------------------------------------------------------------+

\pagestyle{empty}
%\vspace{1cm}
\setlength{\baselineskip}{0.8cm}

%\indent

% +--------------------------------------------------------------------+
% | Enter the text of your abstract below, maximum of 350 words.
% +--------------------------------------------------------------------+

Software engineering projects in industry have been following a trend of using a web based front end framework for interaction between the user and the software. Traditionally, it has been common to use graphical user interface (GUI) frameworks that runs as an application on the users computer. This poses a problem for developers who wish to make their system platform independent and available for a wider range of consumers. Before, developers were left to choose between frameworks like Java’s Swing or GTK and QT for C++ if they wanted a cross platform user interface. However with this web based approach, users are able to open their browser and visit the projects dashboard instead of running an application themselves.

Recently, the front end web framework approach has become a popular solution to creating functional cross platform user interfaces. Products like Metasploit, Openstack, and Puppet Enterprise have all gone towards using front end web frameworks. This approach provides a uniform experience across all platforms, especially since a developer can expect that no matter what platform their users will be on, it will likely have one of the modern browsers to interface with the web frontend framework.

In this paper I present \projectName{}, a front end web framework that interfaces with \ancor{}. \projectName{} aims to provide \ancor{}’s users with an easy to use front end interface for accomplishing various use-cases against the \ancor{} framework. \projectName{} was developed mainly in AngularJS, a lightweight JavaScript framework. This dashboard is able to accomplish everything that the \ancor{} Command Line Interface, or \ancorcli, is able to do. This framework also needed to provide some information about the state of the system through various data gathered from \ancor{} in a human readable format. Not only should it be able to inform the user about the state of \ancor{} but it needs to be able to perform operations against \ancor{} just like the command line interface can do.

This report documents the design and implementation of \projectName. It will detail the necessary background of the project, and overview of the framework, and discussion of implementation and component breakdown. We will also provide an evaluation of the dashboard and a discussion about future work with \projectName{}.
