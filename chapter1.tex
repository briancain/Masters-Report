% +--------------------------------------------------------------------+
% | Chapter 1
% +--------------------------------------------------------------------+

\cleardoublepage

\chapter{Background}
\label{makereference1}

In this chapter, I hope to provide some necessary background information to explain the problem I am solving with \projectName{}. I will also provide some basic information about the \ancor{} framework to interface with. This chapter will also describe some of the potential users of \projectName{}.

\section{Problem Description}
\label{makereference1.2}

Projects in software engineering always provide a way to interact with their application. Traditionally these tools have been in the form of command line interfaces (CLI) or graphical user interfaces (GUI) that run as an application on the users computer or a remote server. CLIs provide a simple and quick way to perform various commands with the provided software, whereas GUIs that run on the users desktop will provide a more visual point-and-click approach to interact with the software.

However within the past few years there has been a trend where developers are instead providing the user with a web interface, or dashboard~\cite{GEN:Few:2006}, for them to interact with the program. This dashboard interface provides a cross-platform experience where developers no longer have to worry about using a GUI framework that will be supported across all major operating systems. Now, developers can create a simple to use frontend framework through the web where the only major differences they have to worry about is between the major browsers. Not only that, but with projects like Bootstrap, cross browser support has never been easier to accomplish as a web application developer. With the creation of these frontend dashboards, users can now interface with programs through their favorite browser without having to install anything extra on their computer.

\projectName{} hopes to continue this trend by providing a front end web dashboard framework that interfaces with \ancor{}. Through modern web technologies like AngularJS. Bootstrap 3, and D3JS, \projectName{} gives users a simple solution to interact with \ancor{}.

\section{\ancor{} Framework}
\label{makereference1.1}

To understand the need for \projectName{}, it will be helpful to explain what \ancor{} ~\cite{DMatrix:Unruh:2014} is and what problem it attempts to solve. This project was developed by the Argus Lab group \cite{Note:ArgusLab:2014} separately as a component of the Moving-Target Defense project.

\ancor{} stands for \emph{Automated eNterprise network COmpileR}. \ancor{} is a cloud automation framework that abstracts the various elements of an enterprise network as defined by a user. This project helps solve the problem when users are interested in using a cloud-based IT system but might not be completely familiar with the lower level details that would be required to set up and deploy. While there are other solutions currently such as IaaS, SaaS or PaaS, these services are not as flexible and depend on what the vendor is interested in providing to their customers. With \ancor{}, users can define a higher level abstraction to construct and manage their cloud-based IT system.

These abstractions are what defines what is called a requirement model. Once this requirement model has been defined, \ancor{} takes those requirements and compiles an enterprise network. Through this technique, \ancor{} is able to model dependencies between the different layers in a given application stack. \ancor{} uses technologies like OpenStack, Puppet, and MCollective to accomplish this.

When a given enterprise network configuration called ANCOR Requirement Model Language (or ARML)~\cite{DMatrix:Unruh:2014} is given to \ancor{}, it will generate all of the required services and define them as instances. In \projectName{}'s case, I will refer to this as a configuration file or configuration specification file. These instances are defined in the specification as roles. A role generally can be defined as something like a webapp, a worker instance, a database master or slave, and so on. These roles will be defined by the user with various attributes for \ancor{} to deal with. Overall, these roles are all related by a goal. The specification has a goal attribute which groups these roles together. An example goal might be an eCommerce website with example roles being a few load balancers for web, some web applications, database master and slaves, and worker nodes.

\section{Users of \projectName}
\label{makereference1.3}

This project was developed with a couple users in mind. Both of these users will be ones who are wanting to interact with \ancor{}, but each user may have different preferences when it comes to interacting with \ancor{}.

Our first example of a user is users who may not be familiar with using the console or operating programs through console commands. In this case, they are not interested in using the \ancorcli{} to interface with \ancor{}. Because of this, they will be more interested in the \projectName{}. \projectName{} offers an alternate solution for these users. This solution is very visual, and gives them a point-and-click experience. The only real typing they might have to do is when they want to deploy a new environment. Even in this case, this configuration might already be defined and they can just paste the configuration file into the \ancor{} configuration editor.

Another example of a potential \projectName{} user is someone who want a more visual experience when interacting with \ancor{}, despite having the skills required to be comfortable in command line. This situation may become more important as scenarios for \ancor{} become more complicated. The \ancorcli{} may end up returning more information than the average person can decypher at once. Using \projectName{} will give users the ability to analyze a large amount of information through graphs and statistics that may be difficult to represent in terminal.

