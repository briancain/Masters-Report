% +--------------------------------------------------------------------+
% | Preface (Optional)
% +--------------------------------------------------------------------+

\newpage
\vspace*{0.9cm}
\begin{center}
{\bf \Huge Preface}
\end{center}

\setlength{\baselineskip}{0.8cm}

%\pdfbookmark[0]{Preface}{PDF_Preface}

% +--------------------------------------------------------------------+
% | Enter text of your Preface in the space below this box.
% +--------------------------------------------------------------------+

Here is my preface.
